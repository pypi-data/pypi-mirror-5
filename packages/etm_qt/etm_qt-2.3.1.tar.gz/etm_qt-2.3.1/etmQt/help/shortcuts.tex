\documentclass[]{article}
\usepackage{amssymb,amsmath}
\usepackage{ifxetex,ifluatex}
\usepackage{fixltx2e} % provides \textsubscript
\ifxetex
  \usepackage{fontspec,xltxtra,xunicode}
  \defaultfontfeatures{Mapping=tex-text,Scale=MatchLowercase}
  \newcommand{\euro}{€}
\else
  \ifluatex
    \usepackage{fontspec}
    \defaultfontfeatures{Mapping=tex-text,Scale=MatchLowercase}
    \newcommand{\euro}{€}
  \else
    \usepackage[utf8]{inputenc}
  \fi
\fi
\usepackage{graphicx}
% We will generate all images so they have a width \maxwidth. This means
% that they will get their normal width if they fit onto the page, but
% are scaled down if they would overflow the margins.
\makeatletter
\def\maxwidth{\ifdim\Gin@nat@width>\linewidth\linewidth
\else\Gin@nat@width\fi}
\makeatother
\let\Oldincludegraphics\includegraphics
\renewcommand{\includegraphics}[1]{\Oldincludegraphics[width=\maxwidth]{#1}}
\ifxetex
  \usepackage[setpagesize=false, % page size defined by xetex
              unicode=false, % unicode breaks when used with xetex
              xetex,
              bookmarks=true,
              pdfauthor={},
              pdftitle={Shortcuts},
              colorlinks=true,
              urlcolor=blue,
              linkcolor=blue]{hyperref}
\else
  \usepackage[unicode=true,
              bookmarks=true,
              pdfauthor={},
              pdftitle={Shortcuts},
              colorlinks=true,
              urlcolor=blue,
              linkcolor=blue]{hyperref}
\fi
\hypersetup{breaklinks=true, pdfborder={0 0 0}}
\setlength{\parindent}{0pt}
\setlength{\parskip}{6pt plus 2pt minus 1pt}
\setlength{\emergencystretch}{3em}  % prevent overfull lines
\setcounter{secnumdepth}{0}

\title{Shortcuts}

\begin{document}
\maketitle

<style>
body {
    margin: auto;
    padding-right: 1em;
    padding-left: 1em;
    max-width: 44em;
    border-left: 1px solid black;
    border-right: 1px solid black;
    color: black;
    font-family: Verdana, sans-serif;
    font-size: 100%;
    line-height: 140%;
    color: #333;
}
pre, tt{
    font-family: monospace;
    background-color:#f8f8f8;
    padding: 2px 4px;
}
code{
    background-color:#f8f8f8;
    white-space: pre-wrap;
    font-size: 110%;
    padding: 1px 1px;
}
h1 a, h2 a, h3 a, h4 a, h5 a, li a {
    text-decoration: none;
    color: #7a5ada;
}
h1, h2, h3, h4, h5 { font-family: verdana;
                     font-weight: bold;
                     /*border-bottom: 1px dotted black;*/
                     color: #7a5ada; }
h1 {
        font-size: 130%;
}

h2 {
        font-size: 110%;
                     border-bottom: 1px dotted black;
}

h3 {
        font-size: 100%;
                     border-bottom: 1px dotted black;
}

h4 {
        font-size: 90%;
        font-style: italic;
                     border-bottom: 1px dotted black;
}

h5 {
        font-size: 85%;
        font-style: italic;
                     border-bottom: 1px dotted black;
}

h1.title {
        font-size: 200%;
        font-weight: bold;
        padding-top: 0.2em;
        padding-bottom: 0.2em;
        text-align: left;
        border: none;
}

dt code {
        font-weight: bold;
}
dd p {
        margin-top: 0;
}

#footer {
        padding-top: 1em;
        font-size: 70%;
        color: gray;
        text-align: center;
}
</style>


\tableofcontents

Note: On Mac OS X, use the \emph{Command} key instead of the
\emph{Control} key.

\subsubsection{General}

\begin{description}
\item[F1]
Show this help information.
\item[F2]
Show information about etm.
\item[F3]
Check for a newer version of etm.
\item[F4]
Display a twelve month calendar. Use left and right cursor keys to
change years and the spacebar to return to the current year.
\end{description}

\includegraphics{images/12monthcalendar.png}
\begin{description}
\item[F5]
Open the datetime calculator.
\end{description}

\includegraphics{images/date_calculator1.png}
\includegraphics{images/date_calculator2.png}
\includegraphics{images/date_calculator3.png}
\includegraphics{images/date_calculator4.png}
\begin{description}
\item[F6]
Show local weather information. (Requires ``weather\_location'' in
\texttt{etm.cfg}.)
\end{description}

\includegraphics{images/weather.png}
\begin{description}
\item[F7]
Show sun and moon data. (Requires ``sunmoon\_location'' in
\texttt{etm.cfg}.)
\end{description}

\includegraphics{images/sunmoon.png}
\begin{description}
\item[SpaceBar]
Display the current date in the day, week and month views. See also
\emph{Control-J} below.
\item[Escape]
Clear the pattern filter and activate the view menu.
\item[Tab]
Toggle the focus between the view menu and the main window.
\item[Control-A]
Show the remaining alerts for today, if any.
\item[Control-C]
If you have a entry for \emph{calendars} in your \emph{etm.cfg} file,
then open a dialog to choose which calendars to display.
\end{description}

\includegraphics{images/calendar_selection.png}
\begin{description}
\item[Control-F]
Enter an expression in the pattern filter to limit the display to items
with matching summaries (titles) or branches. See \emph{Filtering} in
the \emph{Overview} tab.
\item[Control-J]
Enter a fuzzy parsed date to be shown in the day, week and month views.
Relative days and months can be entered in this dialog. E.g., \emph{+21}
to go forward 21 days or \emph{-1/1} to go to the first day of the
previous month. See also \emph{Spacebar} above.
\item[Control-M]
Activate the view menu. You can then select a view using the up and down
arrow keys or the first letter of the view name.
\item[Control-N]
Create a new event, note or task.
\item[Control-O]
Open your \emph{etm.cfg} file for editing.
\item[Control-P]
Switch to the past due view.
\item[Control-R]
Create a custom report.
\item[Control-S]
Open the etm scratch pad.
\item[Control-T]
If the action timer is inactive, create a new action timer. Otherwise
toggle the timer between paused and running.
\item[Shift Control-T]
If the action timer is active, stop the timer and record the action.
\end{description}

\subsubsection{Day View}

\begin{description}
\item[Return]
If a leaf is selected, open the details view for the leaf.
\item[LeftArrow]
Display the last date with scheduled items before the current. Display
the week containing this date in week view and the month containing this
date in month view.
\item[RightArrow]
Display the first date with scheduled items after the current. Display
the week containing this date in week view and the month containing this
date in month view.
\end{description}

\subsubsection{Week View}

\begin{description}
\item[Double-Click]
In a busy time slot, open the details dialog for the relevant event.

In an empty time slot, open a dialog to create a new event for the
relevant date and time.
\item[LeftArrow]
Display the previous week.
\item[RightArrow]
Display the next week.
\item[Control-B]
Open a display showing the periods during the week when you are busy.
\end{description}

\subsubsection{Month View}

\begin{description}
\item[Double-Click]
Make the selected date visible in both the day and week views and switch
to the week view.
\item[LeftArrow]
Move the selection to the previous month.
\item[RightArrow]
Move the selection to the next month.
\item[UpArrow]
Move the selection to the previous week.
\item[DownArrow]
Move the selection to the next week.
\end{description}

\subsubsection{Tree Views}

The following apply to all views other than the week and month views.
Hovering the mouse over a leaf displays a tooltip with the details of
the relevant item.

\begin{description}
\item[Double-Click]
On a branch, toggle between expanded and collapsed.

On a leaf, open the details dialog for the selected item.
\item[Return]
When a leaf is selected, open the details dialog for the selected item.
\item[Control-/]
Open a dialog to choose the level of expansion for the tree.
\item[UpArrow]
Moves the cursor to the item in the same column on the previous row. If
the parent of the current item has no more rows to navigate to, the
cursor moves to the relevant item in the last row of the sibling that
precedes the parent.
\item[DownArrow]
Moves the cursor to the item in the same column on the next row. If the
parent of the current item has no more rows to navigate to, the cursor
moves to the relevant item in the first row of the sibling that follows
the parent.
\item[LeftArrow]
Hides the children of the current item (if present) by collapsing a
branch.
\item[Minus]
Same as LeftArrow.
\item[RightArrow]
Reveals the children of the current item (if present) by expanding a
branch.
\item[Plus]
Same as RightArrow.
\item[Asterisk]
Expands all children of the current item (if present).
\item[PageUp]
Moves the cursor up one page.
\item[PageDown]
Moves the cursor down one page.
\item[Home]
Moves the cursor to an item in the same column of the first row of the
first top-level item in the model.
\item[End]
Moves the cursor to an item in the same column of the last row of the
last top-level item in the model.

\item[Alphabetic and/or numeric character(s)]
Jump to the next appearance of the character(s).
\end{description}

\subsubsection{Details view}

\begin{description}
\item[Return]
Edit this item.
\item[Control-C]
Edit a copy of this item.
\item[Control-D]
Delete this item.
\item[Control-E]
Edit the file containing this item.
\item[Control-F]
If the selected item is a task, enter a finish date for it.
\item[Control-H]
Show the history of changes to the file containing this item.
\item[Control-M]
Move this item to a different file.
\item[Control-R]
If this is a repeating item, show its repetitions.
\item[Control-T]
Start the timer for a new action based on the selected item.
\end{description}

\subsubsection{Reports dialog}

\begin{description}
\item[Escape]
If the list of report specifications is open, close it.
\item[Return]
In the report specification field, add the current specification to the
list if it is not already included. Use Control-S to save such changes
to the list.
\item[Control-D]
Remove the current report specification from the list if it is included.
Use Control-S to save such changes to the list.
\item[Control-E]
Export the current report.
\item[Control-L]
Open the list of report specifications.
\item[Control-P]
Print the current report.
\item[Control-R]
Refresh the report using the selected report options setting.
\item[Control-S]
Save changes to the list of report options settings.
\end{description}

\subsubsection{Editor}

\begin{description}
\item[Control-Return]
Save changes if modified and close the editor.
\item[Control-C]
Copy selection to clipboard.
\item[Control-I]
Insert the contents of the \emph{etm} scratch pad at the cursor
position.
\item[Control-S]
Save changes.
\item[Control-V]
Paste clipboard at cursor position.
\item[Control-W]
Close the editor, prompting to save changes if modified.
\item[Control-X]
Delete selection and copy to clipboard.
\item[Control-Z]
Undo.
\item[Shift Control-Z]
Redo.
\end{description}

\end{document}
