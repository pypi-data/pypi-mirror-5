\documentclass[]{article}
\usepackage{amssymb,amsmath}
\usepackage{ifxetex,ifluatex}
\usepackage{fixltx2e} % provides \textsubscript
\ifxetex
  \usepackage{fontspec,xltxtra,xunicode}
  \defaultfontfeatures{Mapping=tex-text,Scale=MatchLowercase}
  \newcommand{\euro}{€}
\else
  \ifluatex
    \usepackage{fontspec}
    \defaultfontfeatures{Mapping=tex-text,Scale=MatchLowercase}
    \newcommand{\euro}{€}
  \else
    \usepackage[utf8]{inputenc}
  \fi
\fi
\usepackage{graphicx}
% We will generate all images so they have a width \maxwidth. This means
% that they will get their normal width if they fit onto the page, but
% are scaled down if they would overflow the margins.
\makeatletter
\def\maxwidth{\ifdim\Gin@nat@width>\linewidth\linewidth
\else\Gin@nat@width\fi}
\makeatother
\let\Oldincludegraphics\includegraphics
\renewcommand{\includegraphics}[1]{\Oldincludegraphics[width=\maxwidth]{#1}}
\ifxetex
  \usepackage[setpagesize=false, % page size defined by xetex
              unicode=false, % unicode breaks when used with xetex
              xetex,
              bookmarks=true,
              pdfauthor={},
              pdftitle={Data},
              colorlinks=true,
              urlcolor=blue,
              linkcolor=blue]{hyperref}
\else
  \usepackage[unicode=true,
              bookmarks=true,
              pdfauthor={},
              pdftitle={Data},
              colorlinks=true,
              urlcolor=blue,
              linkcolor=blue]{hyperref}
\fi
\hypersetup{breaklinks=true, pdfborder={0 0 0}}
\setlength{\parindent}{0pt}
\setlength{\parskip}{6pt plus 2pt minus 1pt}
\setlength{\emergencystretch}{3em}  % prevent overfull lines
\setcounter{secnumdepth}{0}

\title{Data}

\begin{document}
\maketitle

<style>
body {
    margin: auto;
    padding-right: 1em;
    padding-left: 1em;
    max-width: 44em;
    border-left: 1px solid black;
    border-right: 1px solid black;
    color: black;
    font-family: Verdana, sans-serif;
    font-size: 100%;
    line-height: 140%;
    color: #333;
}
pre, tt{
    font-family: monospace;
    background-color:#f8f8f8;
    padding: 2px 4px;
}
code{
    background-color:#f8f8f8;
    white-space: pre-wrap;
    font-size: 110%;
    padding: 1px 1px;
}
h1 a, h2 a, h3 a, h4 a, h5 a, li a {
    text-decoration: none;
    color: #7a5ada;
}
h1, h2, h3, h4, h5 { font-family: verdana;
                     font-weight: bold;
                     /*border-bottom: 1px dotted black;*/
                     color: #7a5ada; }
h1 {
        font-size: 130%;
}

h2 {
        font-size: 110%;
                     border-bottom: 1px dotted black;
}

h3 {
        font-size: 100%;
                     border-bottom: 1px dotted black;
}

h4 {
        font-size: 90%;
        font-style: italic;
                     border-bottom: 1px dotted black;
}

h5 {
        font-size: 85%;
        font-style: italic;
                     border-bottom: 1px dotted black;
}

h1.title {
        font-size: 200%;
        font-weight: bold;
        padding-top: 0.2em;
        padding-bottom: 0.2em;
        text-align: left;
        border: none;
}

dt code {
        font-weight: bold;
}
dd p {
        margin-top: 0;
}

#footer {
        padding-top: 1em;
        font-size: 70%;
        color: gray;
        text-align: center;
}
</style>


\tableofcontents

\emph{etm} data entries (events, tasks, and so forth) are kept in text
files with the extension \texttt{.txt} in the directory \texttt{datadir}
specified in your \texttt{etm.cfg} file. Items begin with a type
character such as \texttt{*} (event) and continue on one or more lines
either until the end of the file is reached or another line is found
that begins with a type character. The beginning type character for each
item is followed by the item summary and then, perhaps, by one or more
\texttt{@key value} pairs.

The discussion of possible entry types below is followed by a discussion
of the possible \texttt{@key value} pairs.

\subsection{Entry types}

\subsubsection{Action}

A record of the time-consuming action required to complete a task or
participate in an event. Actions are not reminders, they are instead
records of how time was actually spent. Action lines begin with a tilde,
\texttt{\textasciitilde{}}.

\begin{verbatim}
    ~ work on sales presentation @s mon 3p @e 1h15m
\end{verbatim}

Entries such as \texttt{@s mon 3p} and \texttt{@e 1h15m} are discussed
below under \emph{Item details}.

\subsubsection{Event}

Something that will happen on particular day(s) and time(s). Event lines
begin with an asterick, \texttt{*}.

\begin{verbatim}
    * dinner with Karen and Al @s sat 7p @e 3h
\end{verbatim}

Events have a starting datetime, \texttt{@s} and an extent, \texttt{@e}.
The ending datetime is given implicitly as the sum of the starting
datetime and the extent. Events that span more than one day are
possible, e.g.,

\begin{verbatim}
    * Sales meeting @s 9a wed @e 2d8h
\end{verbatim}

begins at 9am on Wednesday and ends at 5pm on Friday.

\subsubsection{All day event}

For holidays, anniversaries, birthdays and the like. Like an event with
a date but no starting time and no extent. All day events begin with a
caret sign, \texttt{\^{}}.

\begin{verbatim}
    ^ The !1776! Independence Day @s 2010-07-04 @r y &M 7 &m 4
\end{verbatim}

On July 4, 2013, this would appear as
\texttt{The 237th Independence Day}.

\subsubsection{Note}

A record of some useful information. Note lines begin with an
exclamation point, \texttt{!}.

\begin{verbatim}
! xyz software @d user: dnlg, pw: abd123gef
\end{verbatim}

\subsubsection{Task}

Something that needs to be done. It may or may not have a due date. Task
lines begin with a minus sign, \texttt{-}.

\begin{verbatim}
- pay bills @s Oct 25
\end{verbatim}

A task with an \texttt{@s} entry becomes due on that date and past due
when that date has passed.

\subsubsection{Delegated task}

A task that is assigned to someone else, usually the person(s)
designated in an \texttt{@u} entry. Delegated tasks begin with a percent
sign, \texttt{\%}.

\begin{verbatim}
    % make reservations for trip @u joe @s fri
\end{verbatim}

\subsubsection{Task group}

A collection of related tasks, some of which may be prerequities for
others. Task groups begin with a plus sign, \texttt{+}.

\begin{verbatim}
    + dog house
      @j pickup lumber      &q 1
      @j cut pieces         &q 2
      @j assemble           &q 3
      @j pickup paint       &q 1
      @j paint              &q 4
\end{verbatim}

Note that a task group is a single item and is treated as such. E.g., if
any job is selected for editing then the entire group is displayed.

Individual jobs are given by the \texttt{@j} entries. The \emph{queue}
entries, \texttt{\&q}, set the order --- tasks with smaller \&q values
are prerequisites for subsequent tasks with larger \&q values. In the
example above, neither ``pickup lumber'' nor ``pickup paint'' have any
prerequisites. ``Pickup lumber'', however, is a prerequisite for ``cut
pieces'' which, in turn, is a prerequisite for ``assemble''. Both
``assemble'' and ``pickup paint'' are prerequisites for ``paint''.

The way a task group is displayed is illustrated below. Note that jobs
are numbered and that jobs with unfinished prerequisites are labeled
with a different icon.
\includegraphics{images/doghouse1.png}
When a job is completed, its icon is changed and it is displayed on the
date the job was completed. Note that completing ``pickup lumber'' makes
``cut pieces'' available for completion.
\includegraphics{images/doghouse2.png}
\subsubsection{In basket}

A quick, don't worry about the details item to be edited later when you
have the time. In basket entries begin with a dollar sign, \texttt{\$}.

\begin{verbatim}
    $ joe 919 123-4567
\end{verbatim}

If you create an item using \emph{etm} and forget to provide a type
character, an \texttt{\$} will automatically be inserted.

\subsubsection{Someday maybe}

Something are you don't want to forget about altogether but don't want
to appear on your next or scheduled lists. Someday maybe items begin
with a question mark, \texttt{?}.

\begin{verbatim}
    ? lose weight and exercise more
\end{verbatim}

\subsubsection{Hidden}

Hidden items begin with a hash mark, \texttt{\#}. Such items are ignored
by etm save for appearing in the folder view. Stick a hash mark in front
of any item that you don't want to delete but don't want to see in your
other views.

\subsubsection{Defaults}

Default entries begin with an equal sign, \texttt{=}. These entries
consist of \texttt{@key value} pairs which then become the defaults for
subsequent entries in the same file until another \texttt{=} entry is
reached.

Suppose, for example, that a particular file contains items relating to
``project\_a'' for ``client\_1''. Then entering

\begin{verbatim}
= @k client_1:project_a
\end{verbatim}

on the first line of the file and

\begin{verbatim}
=
\end{verbatim}

on the twentieth line of the file would set the default keyword for
entries between the first and twentieth line in the file.

\subsection{@key-value pairs}

Note in the following that when the required value in an
\texttt{@key value} pair is a either a \emph{datetime} or an \emph{time
period}, special formats are used.

\emph{etm} supports fuzzy parsing of datetimes. Suppose, for example,
that it is currently Wednesday, November 14, 2012. Then, in
\texttt{@keys} calling for a datetime, values would expand as
illustrated below.

\begin{itemize}
\item
  \texttt{mon 2p}: 2:00pm Monday, November 19
\item
  \texttt{fri}: 12:00am Friday, November 16
\item
  \texttt{9a -1/1}: 9:00am Monday, October 1
\item
  \texttt{+2/15}: 12:00am Tuesday, January 15 2013
\item
  \texttt{8p +7}: 8:00pm Monday, November 26
\item
  \texttt{-14}: 12:00am Monday, November 5
\end{itemize}

In \texttt{@keys} calling for a time period, values would expand as
follows:

\begin{itemize}
\item
  \texttt{2h30m}: 2 hours, 30 minutes
\item
  \texttt{7d}: 7 days
\item
  \texttt{45m} or \texttt{45}: 45 minutes
\end{itemize}

\subsubsection{@a alert}

The specification of the alert(s) to use with the item. One or more
alerts can be specified in an item. E.g.,

\begin{verbatim}
@a 10m, 5m
@a 1h: s
\end{verbatim}

would trigger the alert(s) specified by \texttt{default\_alert} in your
\texttt{etm.cfg} at 10 and 5 minutes before the starting time and a
(s)ound alert one hour before the starting time.

Similary, the alert

\begin{verbatim}
@a 2d: e; who@what.com, where2@when.org; filepath1, filepath2
\end{verbatim}

would send emails to the two listed recipients exactly 2 days (48 hours)
before the starting time of the item with the item summary as the
subject, with file1 and file2 as attachments and with the body of the
message composed using \texttt{email\_template} from your
\texttt{etm.cfg}.

Finally,

\begin{verbatim}
@a 0: p; program_path
\end{verbatim}

would execute \texttt{program\_path} at the starting time of the item.

The format for each of these:

\begin{verbatim}
@a <trigger times> [: action [; arguments]]
\end{verbatim}

In addition to the default action used when the optional
\texttt{: action} is not given, there are five possible values for
\texttt{action}:

\begin{itemize}
\item
  \texttt{d} Execute the setting for \texttt{display} in
  \texttt{etm.cfg}.
\item
  \texttt{v} Execute the setting for \texttt{voice} in the
  \texttt{etm.cfg}.
\item
  \texttt{s} Execute the setting for \texttt{sound} in the
  \texttt{etm.cfg}.
\item
  \texttt{e; recipients{[};attachments{]}} Send an email to
  \texttt{recipients} (a comma separated list of email addresses or a
  template that expands to a list of email addresses) optionally
  attaching \texttt{attachments} (a comma separated list of file paths).
  The item summary is used as the subject of the email and the expanded
  value of \texttt{email\_template} from \texttt{etm.cfg} as the body.
\item
  \texttt{p; process} Execute the command given by \texttt{process}.
\end{itemize}

Note: either \texttt{e} or \texttt{p} can be combined with other actions
in a single alert but not with one another.

\subsubsection{@b beginby}

An integer number of days before the starting date time at which to
begin displaying \emph{begin by} notices.

\subsubsection{@c context}

Intended primarily for tasks to indicate the context in which the task
can be completed. Common contexts include home, office, phone, computer
and errands. The ``next view'' supports this usage by showing undated
tasks, grouped by context. If you're about to run errands, for example,
you can open the ``next view'', look under ``errands'' and be sure that
you will have no ``wish I had remembered'' regrets.

\subsubsection{@d description}

An elaboration of the details of the item to complement the summary.

\subsubsection{@e extent}

A time period string such as \texttt{1d2h} (1 day 2 hours). For an
action, this would be the elapsed time. For a task, this could be an
estimate of the time required for completion. For an event, this would
be the duration. The ending time of the event would be this much later
than the starting datetime.

\subsubsection{@f done; due}

Datetimes; tasks, delegated tasks and task groups only. When a task is
completed an \texttt{@f done; due} entry is added to the task.
Similarly, when a job from a task group is completed in etm, an
\texttt{\&f done; due} entry is appended to the job and it is removed
from the list of prerequisites for the other jobs. In both cases
\texttt{done} is the completion datetime and \texttt{due} is the
datetime that the task or job was due. The completed task or job is
shown as finished on the completion date. When the last job in a task
group is finished an \texttt{@f done; due} entry is added to the task
group itself reflecting the datetime that the last job was done and, if
the task group is repeating, the \texttt{\&f} entries are removed from
the individual jobs.

Another step is taken for repeating task groups. When the first job in a
task group is completed, the \texttt{@s} entry is updated using the
setting for \texttt{@o} (above) to show the next datetime the task group
is due and the \texttt{@f} entry is removed from the task group. This
means when some, but not all of the jobs for the current repetition have
been completed, only these job completions will be displayed. Otherwise,
when none of the jobs for the current repetition have been completed,
then only that last completion of the task group itself will be
displayed.

Consider, for example, the following repeating task group which repeats
monthly on the last weekday on or before the 25th.

\begin{verbatim}
+ pay bills @s 11/23 @f 10/24;10/25
  @r m &w MO,TU,WE,TH,FR &m 23,24,25 &s -1
  @j organize bills &q 1
  @j pay on-line bills &q 3
  @j get stamps, envelopes, checkbook &q 1
  @j write checks &q 2
  @j mail checks &q 3
\end{verbatim}

Here ``organize bills'' and ``get stamps, envelopes, checkbook'' have no
prerequisites. ``Organize bills'', however, is a prerequisite for ``pay
on-line bills'' and both ``organize bills'' and ``get stamps, envelops,
checkbook'' are prerequisites for ``write checks'' which, in turn, is a
prerequisite for ``mail checks''.

The repetition that was due on 10/25 was completed on 10/24. The next
repetition was due on 11/23 and, since none of the jobs for this
repetition have been completed, the completion of the group on 10/24 and
the list of jobs due on 11/23 will be displayed initially. The following
sequence of screen shots show the effect of completing the jobs for the
11/23 repetition one by one on 11/27.

\includegraphics{images/paybills1.png}
\includegraphics{images/paybills2.png}
\includegraphics{images/paybills3.png}
\includegraphics{images/paybills4.png}
\includegraphics{images/paybills5.png}
\includegraphics{images/paybills6.png}
\subsubsection{@g goto}

The path to a file or a URL to be opened using the system default
application when the user presses \emph{Control-G} in the GUI.

\subsubsection{@j job}

Component tasks or jobs within a task group are given by \texttt{@j job}
entries. \texttt{@key value} entries prior to the first \texttt{@j}
become the defaults for the jobs that follow. \texttt{\&key value}
entries given in jobs use \texttt{\&} rather than \texttt{@} and apply
only to the specific job.

Many key-value pairs can be given either in the group task using
\texttt{@} or in the component jobs using \texttt{\&}:

\begin{itemize}
\item
  \texttt{@c} or \texttt{\&c}: context
\item
  \texttt{@d} or \texttt{\&d}: description
\item
  \texttt{@e} or \texttt{\&e}: extent
\item
  \texttt{@f} or \texttt{\&f}: done; due (datetimes)
\item
  \texttt{@k} or \texttt{\&k}: keyword
\item
  \texttt{@l} or \texttt{\&l}: location
\end{itemize}

The key-value pair \texttt{\&q} (queue position) can \emph{only} be
given in component jobs where it is required. Key-values other than
\texttt{\&q} and those listed above, can \emph{only} be given in the
initial group task entry and their values are inherited by the component
jobs.

\subsubsection{@k keyword}

A heirarchical classifier for the item. Intended for actions to support
time billing where a common format would be
\texttt{client:job:category}. \emph{etm} treats such a keyword as a
heirarchy so that an action report grouped by month and then keyword
might appear as follows

\begin{verbatim}
    27.5h) Client 1 (3)
        4.9h) Project A (1)
        15h) Project B (1)
        7.6h) Project C (1)
    24.2h) Client 2 (3)
        3.1h) Project D (1)
        21.1h) Project E (2)
            5.1h) Category a (1)
            16h) Category b (1)
    4.2h) Client 3 (1)
    8.7h) Client 4 (2)
        2.1h) Project F (1)
        6.6h) Project G (1)
\end{verbatim}

An arbitrary number of heirarchical levels in keywords is supported.

\subsubsection{@l location}

The location at which, for example, an event will take place.

\subsubsection{@m memo}

Further information about the item not included in the summary or the
description. Since the summary is used as the subject of an email alert
and the descripton is commonly included in the body of an email alert,
this field could be used for information not to be included in the
email.

\subsubsection{@o overdue}

Repeating tasks only. One of the following choices: k) keep, r) restart,
or s) skip. Details below.

\subsubsection{@p priority}

Either 0 (no priority) or an intger between 1 (highest priority) and 9
(lowest priority). Primarily used with undated tasks.

\subsubsection{@r repetition rule}

The specification of how an item is to repeat. Repeating items
\textbf{must} have an \texttt{@s} entry as well as one or more
\texttt{@r} entries. Generated datetimes are those satisfying any of the
\texttt{@r} entries and falling \textbf{on or after} the datetime given
in \texttt{@s}.

A repetition rule begins with

\begin{verbatim}
@r frequency
\end{verbatim}

where \texttt{frequency} is one of the following characters:

\begin{itemize}
\item
  \texttt{y}: yearly
\item
  \texttt{m}: monthly
\item
  \texttt{m}: weekly
\item
  \texttt{d}: daily
\item
  \texttt{l}: list (a list of datetimes will be provided using
  \texttt{\&+})
\end{itemize}

The \texttt{@r frequency} entry can, optionally, be followed by one or
more \texttt{\&key value} pairs:

\begin{itemize}
\item
  \texttt{\&i}: interval (positive integer, default = 1) E.g, with
  frequency \texttt{w}, interval 3 would repeat every three weeks.
\item
  \texttt{\&t}: total (positive integer) Include no more than this total
  number of repetitions.
\item
  \texttt{\&s}: bysetpos (integer) See the payday example below for an
  illustration of bysetpos.
\item
  \texttt{\&u}: until (datetime) Only include repetitions falling
  \emph{before} (not including) this datetime.
\item
  \texttt{\&M}: bymonth (1, 2, \ldots{}, 12)
\item
  \texttt{\&m}: bymonthday (1, 2, \ldots{}, 31)
\item
  \texttt{\&W}: byweekno (1, 2, \ldots{}, 53)
\item
  \texttt{\&w}: byweekday (integer 0, 1, \ldots{}, 6 or English weekday
  abbreviation SU \ldots{} SA)
\item
  \texttt{\&h}: byhour (0 \ldots{} 23)
\item
  \texttt{\&n}: byminute (0 \ldots{} 59)
\end{itemize}

Repetition examples:

\begin{itemize}
\item
  Payday (an all day event) on the last week day of each month. (The
  \texttt{\&s -1} entry extracts the last date which is both a weekday
  and falls within the last three days of the month.)

\begin{verbatim}
^ payday @s 2010-07-01
  @r m &w MO, TU, WE, TH, FR &m -1, -2, -3 &s -1
\end{verbatim}
\item
  Take a prescribed medication daily (an event) from the 23rd through
  the 27th of the current month at 10am, 2pm, 6pm and 10pm and trigger
  an alert zero minutes before each event.

\begin{verbatim}
* take Rx @d 10a 23  @r d &u 11p 27 &h 10, 14 18, 22 @a 0
\end{verbatim}
\item
  Vote for president (an all day event) every four years on the first
  Tuesday after a Monday in November. (The \texttt{\&m range(2,9)}
  requires the month day to fall within 2 \ldots{} 8 and thus, combined
  with \texttt{\&w TU} to be the first Tuesday following a Monday.)

\begin{verbatim}
^ Vote for president @s 2012-11-06
  @r y &i 4 &M 11 &m range(2,9) &w TU
\end{verbatim}
\end{itemize}

Optionally, \texttt{@+} and \texttt{@-} entries can be given.

\begin{itemize}
\item
  \texttt{@+}: include (comma separated list to datetimes to be
  \emph{added} to those generated by the \texttt{@r} entries)
\item
  \texttt{@-}: exclude (comma separated list to datetimes to be
  \emph{removed} from those generated by the \texttt{@r} entries)
\end{itemize}

A repeating \emph{task} may optionally also include an
\texttt{@o \textless{}k\textbar{}s\textbar{}r\textgreater{}} entry
(default = k):

\begin{itemize}
\item
  \texttt{@o k}: Keep the current due date if it becomes overdue and use
  the next due date from the recurrence rule if it is finished early.
  This would be appropriate, for example, for the task `file tax
  return'. The return due April 15, 2009 must still be filed even if it
  is overdue and the 2010 return won't be due until April 15, 2010 even
  if the 2009 return is finished early.
\item
  \texttt{@o s}: Skip overdue due dates and set the due date for the
  next repetition to the first due date from the recurrence rule on or
  after the current date. This would be appropriate, for example, for
  the task `put out the trash' since there is no point in putting it out
  on Tuesday if it's picked up on Mondays. You might just as well wait
  until the next Monday to put it out. There's also no point in being
  reminded until the next Monday.
\item
  \texttt{@o r}: Restart the repetitions based on the last completion
  date. Suppose you want to mow the grass once every ten days and that
  when you mowed yesterday, you were already nine days past due. Then
  you want the next due date to be ten days from yesterday and not
  today. Similarly, if you were one day early when you mowed yesterday,
  then you would want the next due date to be ten days from yesterday
  and not ten days from today.
\end{itemize}

\subsubsection{@s starting datetime}

When an action is started, an event begins or a task is due.

\subsubsection{@t tags}

A tag or list of tags for the item.

\subsubsection{@u user}

Intended to specify the person to whom a delegated task is assigned.
Could also be used in actions to indicate the person performing the
action.

\subsubsection{@v value}

A key from \texttt{values} in your \texttt{etm.cfg}. Used in actions to
apply a billing rate to time spent in an action. E.g., with

\begin{verbatim}
    values:
        br1: 45.0
        br2: 60.0
\end{verbatim}

then entries of \texttt{@v br1} and \texttt{@e 2h30m} in an action would
entail a value of \texttt{45.0 * 2.5 = 112.50}.

\subsubsection{@z time zone}

The time zone of the item, e.g., US/Eastern. The starting and other
datetimes in the item will be interpreted as belonging to this time
zone.

\subsubsection{@+ include}

A datetime or list of datetimes to be added to the repetitions generated
by the \texttt{@r rrule} entry.

\subsubsection{@- exclude}

A datetime or list of datetimes to be removed from the repetitions
generated by the \texttt{@r rrule} entry.

\subsection{File paths}

\emph{etm} offers two heirarchical ways of organizing your data: by file
path and by keyword. There are no hard and fast rules about how to use
these heirarchies but the goal is a system that makes complementary uses
of file path and keyword and fits your needs. As with any filing system,
planning and consistency are paramount.

For example, one pattern of use for a business would be to use file path
for people and keyword for client-project-category.

Similarly, a family could use file path to separate personal and shared
items for family members, for example:

\begin{verbatim}
root etm data directory
    dag
    erp
    shared
        holidays
        birthdays
        events
\end{verbatim}

Here

\begin{verbatim}
~dag/.etm/etm.cfg
~erp/.etm/etm.cfg
\end{verbatim}

would both contain \texttt{datadir} entries specifying the common root
data directory. Additionally, if these configuration files contained,
respectively, the entries

\begin{verbatim}
calendars
- [dag, true, dag]
- [erp, false, erp]
- [shared, true, shared]
\end{verbatim}

and

\begin{verbatim}
calendars
- [erp, true, erp]
- [dag, false, dag]
- [shared, true, shared]
\end{verbatim}

then, by default, both dag and erp would see the entries from their
personal files as well as the shared entries and each could optionally
view the entries from the other's personal files as well. See the
\emph{Preferences} tab for details on the \texttt{calendars} entry.

\end{document}
