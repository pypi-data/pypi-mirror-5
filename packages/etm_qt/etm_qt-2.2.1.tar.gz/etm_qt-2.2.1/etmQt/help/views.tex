\documentclass[]{article}
\usepackage{amssymb,amsmath}
\usepackage{ifxetex,ifluatex}
\usepackage{fixltx2e} % provides \textsubscript
\ifxetex
  \usepackage{fontspec,xltxtra,xunicode}
  \defaultfontfeatures{Mapping=tex-text,Scale=MatchLowercase}
  \newcommand{\euro}{€}
\else
  \ifluatex
    \usepackage{fontspec}
    \defaultfontfeatures{Mapping=tex-text,Scale=MatchLowercase}
    \newcommand{\euro}{€}
  \else
    \usepackage[utf8]{inputenc}
  \fi
\fi
\usepackage{graphicx}
% We will generate all images so they have a width \maxwidth. This means
% that they will get their normal width if they fit onto the page, but
% are scaled down if they would overflow the margins.
\makeatletter
\def\maxwidth{\ifdim\Gin@nat@width>\linewidth\linewidth
\else\Gin@nat@width\fi}
\makeatother
\let\Oldincludegraphics\includegraphics
\renewcommand{\includegraphics}[1]{\Oldincludegraphics[width=\maxwidth]{#1}}
\ifxetex
  \usepackage[setpagesize=false, % page size defined by xetex
              unicode=false, % unicode breaks when used with xetex
              xetex,
              bookmarks=true,
              pdfauthor={},
              pdftitle={Views},
              colorlinks=true,
              urlcolor=blue,
              linkcolor=blue]{hyperref}
\else
  \usepackage[unicode=true,
              bookmarks=true,
              pdfauthor={},
              pdftitle={Views},
              colorlinks=true,
              urlcolor=blue,
              linkcolor=blue]{hyperref}
\fi
\hypersetup{breaklinks=true, pdfborder={0 0 0}}
\setlength{\parindent}{0pt}
\setlength{\parskip}{6pt plus 2pt minus 1pt}
\setlength{\emergencystretch}{3em}  % prevent overfull lines
\setcounter{secnumdepth}{0}

\title{Views}

\begin{document}
\maketitle

<style>
body {
    margin: auto;
    padding-right: 1em;
    padding-left: 1em;
    max-width: 44em;
    border-left: 1px solid black;
    border-right: 1px solid black;
    color: black;
    font-family: Verdana, sans-serif;
    font-size: 100%;
    line-height: 140%;
    color: #333;
}
pre, tt{
    font-family: monospace;
    background-color:#f8f8f8;
    padding: 2px 4px;
}
code{
    background-color:#f8f8f8;
    white-space: pre-wrap;
    font-size: 110%;
    padding: 1px 1px;
}
h1 a, h2 a, h3 a, h4 a, h5 a, li a {
    text-decoration: none;
    color: #7a5ada;
}
h1, h2, h3, h4, h5 { font-family: verdana;
                     font-weight: bold;
                     /*border-bottom: 1px dotted black;*/
                     color: #7a5ada; }
h1 {
        font-size: 130%;
}

h2 {
        font-size: 110%;
                     border-bottom: 1px dotted black;
}

h3 {
        font-size: 100%;
                     border-bottom: 1px dotted black;
}

h4 {
        font-size: 90%;
        font-style: italic;
                     border-bottom: 1px dotted black;
}

h5 {
        font-size: 85%;
        font-style: italic;
                     border-bottom: 1px dotted black;
}

h1.title {
        font-size: 200%;
        font-weight: bold;
        padding-top: 0.2em;
        padding-bottom: 0.2em;
        text-align: left;
        border: none;
}

dt code {
        font-weight: bold;
}
dd p {
        margin-top: 0;
}

#footer {
        padding-top: 1em;
        font-size: 70%;
        color: gray;
        text-align: center;
}
</style>


\tableofcontents

\subsubsection{Day}

All scheduled (dated) items appear in this view, grouped by date and
sorted by starting time and item type. Hovering the mouse over an item
brings up a tooltip with details.

\includegraphics{images/day_view.png}
Double clicking on an item or pressing \emph{Return} with an item
selected opens a details dialog for the item with further options. Press
F1 when the details dialog is active for help information on that view.

\includegraphics{images/details_view.png}
Begin by warnings for upcoming events and tasks also appear in this view
on the current date when the current date falls within the range of the
begin by warning.

\subsubsection{Week}

A graphical view of a week showing scheduled events and free periods.
Hovering over a busy period brings up a tooltip with the details of the
event.

\includegraphics{images/week_view.png}
Double clicking on a busy period opens the details dialog for the item.
Double clicking on a free period opens a dialog to create a new event
for that date and time.

This view conforms to the iso standard in which weeks begin on Monday
and the first week in a year is the week that contains the first
Thursday. This first week is assigned number 1 and the last week number
in the year will then either be 52 or 53. E.g., in 2015, week number 1
is December 29, 2014 --- January 4, 2015 and week number 53 is December
28, 2015 --- January 3, 2016.

If you have an entry for \texttt{calendars} in your \texttt{etm.cfg},
then busy times from your default calendars will appear in one color and
busy times from your non-default calendars will appear in another color.

Need to tell someone when you are available during a given week? Select
the week in this view, press Control-B to display the periods during the
week when you are busy and then copy and paste these into an email.

\includegraphics{images/busy_times.png}
\subsubsection{Month}

A monthly calendar view with the date numbers colored to indicate the
amount of scheduled time for events for that date.

\includegraphics{images/month_view.png}
Double clicking on a date, switches to the week view for that date and
scrolls the day view to that date as well. Month view also conforms to
the iso standard in which weeks begin on Monday.

\subsubsection{Past Due}

All dated tasks whose due dates have passed including delegated tasks
and waiting tasks (tasks with unfinished prerequisites) grouped by
available, delegated and waiting and, within each group, by the due
date.

When there are past due tasks, a red warning button appears in the
lower, right-hand corner of the main display --- see, for example, the
week and month view screen shots above. This button disappears when
there are no past due tasks.

\subsubsection{Next}

All \emph{unscheduled} (undated) tasks grouped by context (home, office,
phone, computer, errands and so forth) and sorted by priority. These
tasks correspond to GTD's \emph{next actions}. These are tasks which
don't really have a deadline and can be completed whenever a convenient
opportunity arises. Check this view, for example, before you leave to
run errands for opportunities to clear other errands.

Items without contexts are automatically assigned the context ``none''.

``In basket'' and ``someday maybe'' items are also displayed in this
view.

\subsubsection{Folder}

All items grouped by folder (project file path) and sorted by type and
\emph{relevant datetime}. Use this view to review the status of your
projects.

The \emph{relevant datetime} is the past due date for any past due
tasks, the starting datetime for any non-repeating items and the
datetime of the next instance for any repeating items.

\subsubsection{Keyword}

All items grouped by keyword and sorted by type and \emph{relevant
datetime}.

Items without keywords are automatically assigned the keyword ``none''.

\subsubsection{Tag}

All items with tag entries grouped by tag and sorted by type and
\emph{relevant datetime}. Note that items with multiple tags will be
listed under each tag.

\end{document}
