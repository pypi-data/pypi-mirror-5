\newcommand{\this}{paper}

\documentclass{article}

\title{\Huge Targeted characterisation of short structural variation.}
\author{S.Y. Anvar, K.J. van der Gaag, J.W.F. van der Heijden,\\
  R.H.A.M. Vossen, R.H. de Leeuw, M.H.A.M. Veltrop,\\
  H.P.J. Buermans, J.S. Verbeek, J.T. den Dunnen, P. de Knijff,\\
  J.F.J. Laros
  \vspace{10pt}\\
  Department of Human Genetics\\
  Center for Human and Clinical Genetics\\
  \texttt{j.f.j.laros@lumc.nl}}
\date{\today}

%\setlength{\parindent}{0pt}
\frenchspacing

\begin{document}

\maketitle

\begin{abstract} \noindent
\end{abstract}

\section{Introduction}\label{introduction}

\section{Materials and Methods}
We developed a
program\footnote{https://humgenprojects.lumc.nl/trac/dna-profile/} that
implements the characterisation of short structural variation. In this section,
we describe the functionality and design of this program. Calibration of the
algorithms and output is done with optional command line arguments described in
Table~\ref{tab:args}.

\subsection{Input} \label{subsec:input}
Our method expects two input files: one file containing sequencing data in
\texttt{fasta} format and one file containing the library description. The
format of this description is shown in Table~\ref{tab:library}. The last column
of the description is a user friendly way of supplying a simple form of
\emph{regular expressions}. It consists of a series of triples, each containing
a sequence followed by two integers, which denote the minimum and maximum
number of times the preceding sequence is expected. This notation of expected
alleles is compiled to a regular expression, which is used to distinguish
between known and unknown alleles.

\subsection{Marker alignment} \label{subsec:align}
Each pair of markers is aligned to each read by using a semi-global pairwise
alignment, a modified version of the Smith-Waterman~\cite{LA1} algorithm. In
this adaptation the alignment matrix is initialised with penalties for the
aligned sequence, but not for the reference sequence. By using this approach,
we can use the alignment matrix for the calculation of the minimum edit
distance~\cite{Lev} between the aligned sequence and all substrings of the
reference sequence. Finally, we use the alignment matrix to select the
rightmost alignment with a minimum edit distance. To guarantee that this method
is symmetrical with regard to \emph{reverse complement}, we align the reverse
complement of the right marker to the reverse complement of the reference
sequence.

\subsection{Allele identification}
If a marker pair can be aligned to either the forward or reverse complement of
the reference sequence, we can select the area of interest by extracting the
sequence between the alignment coordinates (see Section~\ref{subsec:align}) and
by converting the selection to the forward orientation. This area of interest
can then be matched to the regular expression (see Section~\ref{subsec:input})
of that marker pair. Depending on the match, we either classify the area of
interest as a known or new allele.

\subsection{Optional classification of the input} \label{subsec:class}
Apart from identification and quantification of known and new alleles, the
program can also classify the sequencing data itself based on the alignment of
the marker pairs. If the allele identification is successful, the input
sequence in question is classified as either known, or new. If the
identification is only partially successful, i.e., only the left or the right
marker of the pair is identified, the input sequence is classified as having no
beginning or no end.

Based on this classification, overview tables (see
Table~\ref{tab:overview}), marker specific partitions of the input data (see
Table~\ref{tab:output}) and a file containing all unclassified reads can be
made.

\subsection{Output}
The output of the analysis consists of an overview report that contains a
general overview (total number of reads, number of matched pairs, number of
newly identified alleles, etc.), an overview of the marker pair alignments
(see Table~\ref{tab:marker}) and per marker pair a detailed overview of the
identified alleles, both the expected ones and the newly identified ones (see
Table~\ref{tab:allele}). For the expected alleles, the sequence of the allele
is summarised according to the allele definition in the library.

If an output directory is selected, a folder with this name will be created
containing general overview tables and a \texttt{fasta} file containing
unrecognised sequencing data (see Table~\ref{tab:overview})

Per marker a subfolder is created containing the new alleles and split
\texttt{fasta} files as described in Section~\ref{subsec:class}. The contents
of this subfolder is shown in Table~\ref{tab:output}.

\section{Discussion}

\section{Conclusions and further research}

\bibliography{$HOME/projects/bibliography}{}
\bibliographystyle{plain}

\appendix

\section{Tables}
\begin{table}[h]
  \caption{Library definition.}
  \label{tab:library}
  \begin{center}
    \begin{tabular}{l|l}
      column & description\\
      \hline
      $1$ & Name of the marker pair.\\
      $2$ & Sequence of the left marker.\\
      $3$ & Sequence of the right marker.\\
      $4$ & Description of the expected alleles.\\
    \end{tabular}
  \end{center}
\end{table}

\begin{table}[]
  \caption{Command line optional arguments.}
  \label{tab:args}
  \begin{center}
    \begin{tabular}{l|c|l}
      name & default & description\\
      \hline
      \texttt{-h} &        & Show a description of the usage.\\
      \texttt{-m} & 0.08   & Number of mismatches per nucleotide.\\
      \texttt{-r} & stdout & Name of the report file.\\
      \texttt{-d} &        & Name of the output directory.\\
      \texttt{-a} & 0      & Minimum count per allele.\\
    \end{tabular}
  \end{center}
\end{table}

\begin{table}[h]
  \caption{Generated overview files.}
  \label{tab:overview}
  \begin{center}
    \begin{tabular}{l|l}
      name & description\\
      \hline
      \texttt{unknown.fa}       & Reads classified as unknown.\\
      \texttt{summary.csv}      & Global overview of total reads, matched
        pairs, etc.\\
      \texttt{markers.csv}      & General overview of marker pair alignments
        (see Table~\ref{tab:marker}).\\
      \texttt{knownalleles.csv} & Overview of all known alleles (see
        Table~\ref{tab:knownnew}).\\
      \texttt{newalleles.csv}   & Overview of all new alleles (see
        Table~\ref{tab:knownnew}).\\
      \texttt{nostart.csv}      & Overview of single right marker matches (see
        Table~\ref{tab:nostartend}).\\
      \texttt{noend.csv}        & Overview of single left marker matches (see
        Table~\ref{tab:nostartend}).\\
    \end{tabular}
  \end{center}
\end{table}

\begin{table}[h]
  \caption{Structure of the known and new allele summary files.}
  \label{tab:knownnew}
  \begin{center}
    \begin{tabular}{l|l}
      name & description\\
      \hline
      name    & Name of the marker pair.\\
      forward & Number of pair matches in forward orientation.\\
      reverse & Number of pair matches in reverse orientation.\\
      total   & Total number of pair matches.\\
      allele  & Sequence of the area of interest.\\
    \end{tabular}
  \end{center}
\end{table}

\begin{table}[h]
  \caption{Structure of the no start- or end summary files.}
  \label{tab:nostartend}
  \begin{center}
    \begin{tabular}{l|l}
      name & description\\
      \hline
      name    & Name of the marker pair.\\
      forward & Number of single marker matches in forward orientation.\\
      reverse & Number of single marker matches in reverse orientation.\\
      total   & Total number single marker matches.\\
    \end{tabular}
  \end{center}
\end{table}

\begin{table}[h]
  \caption{Generated output files per marker.}
  \label{tab:output}
  \begin{center}
    \begin{tabular}{l|l}
      name & description\\
      \hline
      \texttt{known.fa}         & Reads classified as known.\\
      \texttt{new.fa}           & Reads classified as new.\\
      \texttt{nostart.fa}       & Reads where only the left marker was found.\\
      \texttt{noend.fa}         & Reads where only the right marker was
        found.\\
      \texttt{knownalleles.csv} & Table of known alleles (see
        Table~\ref{tab:allele}).\\
      \texttt{newalleles.csv}   & Table of new alleles (see
        Table~\ref{tab:allele}).\\
    \end{tabular}
  \end{center}
\end{table}

\begin{table}[h]
  \caption{Marker overview.}
  \label{tab:marker}
  \begin{center}
    \begin{tabular}{l|l}
      name & description\\
      \hline
      name    & Name of the marker pair.\\
      fPaired & Number of pair matches in forward orientation.\\
      rPaired & Number of pair matches in reverse orientation.\\
      fLeft   & Number of left marker matches in forward orientation.\\
      rLeft   & Number of left marker matches in reverse orientation.\\
      fRight  & Number of right marker matches in forward orientation.\\
      rRight  & Number of right marker matches in reverse orientation.\\
    \end{tabular}
  \end{center}
\end{table}

\begin{table}[h]
  \caption{Allele overview.}
  \label{tab:allele}
  \begin{center}
    \begin{tabular}{l|l}
      name & description\\
      \hline
      allele  & Sequence of the area of interest.\\
      total   & Number of times the allele was found.\\
      forward & Number of times found in forward orientation.\\
      reverse & Number of times found in reverse orientation.\\
    \end{tabular}
  \end{center}
\end{table}

\end{document}
